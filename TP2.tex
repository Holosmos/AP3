\documentclass{mybourbaki}
\titre{TP2 : Suite de \textsc{Fibonacci}}
\author{Raphaël Alexandre}
\lstset{language=Python}


\begin{document}
\begin{center}
Raphaël \textsc{Alexandre}
\end{center}

\bigbreak
\bigbreak

\section*{Suite de \textsc{Fibonacci}}

Une suite, $(F_n)_{n\in \N}$, de \textsc{Fibonacci} est décrite par la relation de récurrence : \[ \forall n \in \N, \; F_{n+2} = F_{n+1} + F_n.\]
On prendra pour conditions initiales dans la suite : \[ F_0 = 0 \; ; \;  F_1 = 1.\]

\section{Algorithme naïf}
On écrit un premier algorithme naïf à partir de la définition de la suite par la relation de récurrence.

\begin{lstlisting}[frame=single] 
def fib1(n):
	if n <2:
		return n
	return fib1(n-1)+fib1(n-2) 
\end{lstlisting}

Soit $n\in \N$, établissons le nombre d'appels, $A(n)$, à $F_1$ pour le calcul de $F_n$ par cet algorithme :
\begin{itemize}
\item pour $n=0$, $A(0) = 0$ et pour $n=1$, $A(1) = 1$ ;
\item soit $n\geq 2$ fixé, $A(n+2) = A(n+1) + A(n)$ d'après l'appel récursif.
\end{itemize}
C'est donc aussi une suite de \textsc{Fibonacci} et les conditions initiales sont encore une fois $0$ et $1$. On a donc un coût exponentiel selon $n$.

Si on s'intéresse au nombre d'appels, $A_k(n)$ à la valeur $F_k$ pour $k<n$ fixé on a :
\begin{itemize}
\item $A_k(p) = 0$ pour $p<k$, $A_k(k) = 1$ ;
\item $A_k(n+2) = A_k(n+1) + A_k(n)$.
\end{itemize}
On en déduit naturellement $A_k(n) = A(n-k+1)$ pour $n>k$ (on peut naturellement l'étendre à $n\geq k$).

Une mesure pour évaluer le temps d'exécution de l'algorithme pour $F(18)$ et $F(30)$ donne :
\begin{lstlisting}[frame = single]
print time.clock()
print fib1(18)
print time.clock()
print fib1(30)
print time.clock()
\end{lstlisting}
\begin{lstlisting}[frame=single,language=bash]
air-de-raphael:Info Raphael$ python TP2.py
0.020602
2584
0.022091
832040
0.377324
\end{lstlisting}

On mesure donc un temps de $0.001489$ secondes pour $F(18)$ et pour $F(30)$ : $0.355233$ secondes.

\section{Algorithme linéaire}
L'écriture de ce second algorithme repose sur \og l'astuce \fg{} de garder en mémoire les termes déjà calculés. On fait donc suivre d'un appel récursif à l'autre la liste grandissante des $F_0,F_1,\ldots,F_k$ termes.

\begin{lstlisting}[frame = single]
def fib2(n,liste = [0,1]):
	a = len(liste)
	if n == a-1 or n == a-2:
		return liste[n]
	liste.append(liste[a-1]+liste[a-2])
	return fib2(n,liste)
\end{lstlisting}

Le passage à la récursivité se fait en deux étapes : on allonge la liste avec le terme nouvellement calculé avec les deux derniers et on effectue l'appel récursif.

Calculons le nombre d'appels récursifs, $A(n)$ :
\begin{itemize}
\item pour $n=0,1$ il est clair que le nombre d'appel est de $0$ : le retour est immédiat ;
\item pour $n> 1$, on a la relation : $A(n) = 1 + A(n-1)$.
\end{itemize}
On a clairement $A(n) = n-1$ pour $n> 1$ et $A(0) = A(1) = 0$. C'est bien une progression linéaire.

Mesurons le temps pour $F(18)$ et $F(30)$ :
\begin{lstlisting}[frame = single]
print time.clock()
print fib2(18)
print time.clock()
print fib2(30)
print time.clock()
\end{lstlisting}
\begin{lstlisting}[frame=single,language=bash]
air-de-raphael:Info Raphael$ python TP2.py
0.017101
2584
0.017151
832040
0.017162
\end{lstlisting}
Soit des durées de $0.000051$ secondes et $0.000011$ secondes. Cependant à ce niveau ce n'est plus très représentatif (selon les divers processus en exécutions parallèles).

\section{Algorithme logarithmique}
Pour ce dernier algorithme qui s'avèrera le plus efficace des trois, on utilise les relations suivantes :
\begin{align*}
\forall k\geq 1, \; F_{2k} &= (2F_{k-1}+F_k)F_k \\
\forall k\geq 1, \; F_{2k+1} &= F_{k+1}^{2} + F_k^{2}
\end{align*}
Un premier algorithme \og naïf \fg{} avec ces informations supplémentaires serait :

\begin{lstlisting}[frame = single]
def fib3(n):
	if n<2:
		return n
	fk = fib3(n/2)
	if n%2==0:
		return (2*fib3(n/2-1)+fk)*fk
	return fib3(n/2+1)**2+fk**2	
\end{lstlisting}

Cependant les appels récursifs s'accumulent du fait de la nécessité d'au moins deux termes pour calculer. On utilise donc une autre astuce qui est de transmettre un couple $(F_{k-1},F_k)$ à chaque itération pour économiser ces doubles appels.

On construit donc l'algorithme :
\begin{lstlisting}[frame = single]
def fib3Aux(n):
	if n<2:
		return (0,1)
	a = fib3Aux(n/2)
	if n%2==0:	
		return (a[0]**2+a[1]**2,(2*a[0]+a[1])*a[1])
	b = a[0]+a[1]
	return ((2*a[0]+a[1])*a[1],b**2+a[1]**2)
\end{lstlisting}
Décrivons un peu plus la manière dont l'algorithme calcule le terme $F(n)$.

Soit $n\geq 2$ (le cas $n=0 \ou 1$ étant aisé à comprendre) :
\begin{enumerate}
\item On fait un appel récursif pour récupérer le couple $(F(n\div 2 -1),F(n \div 2))$ où $n \div 2$ est la division entière de $n$ par $2$. Dans le code : a[$0$]$=F(n\div 2 -1)$ (notons ce nombre $x$) et a[$1$]$=F(n\div 2)$ (notons ce nombre $y$).
\item Si $n$ est pair, c'est-à-dire : $n = 2k$ avec $k\geq 1$. Alors $x = F(k-1)$ et $y = F(k)$. Par la relation on a : \[(F(n-1),F(n)) = (F(2k -1),F(2k)) = (x^2 + y^2, (2x+y)y)\]ce qui justifie cette partie du code.
\item Enfin, si $n$ est impair, c'est-à-dire $n=2k+1$ avec $k\geq 1$. On a $x = F(k-1)$ et $y=F(k)$. Alors on calcule tout d'abord $z=x+y = F(k +1)$. On calcule ensuite : \[(F(n-1),F(n)) = (F(2k),F(2k+1)) = ((2x+y)y,z^2+y^2)\]ce qui justifie la fin du code.
\end{enumerate}

Remarquons que cet algorithme renvoie le couple $(F(n-1),F(n))$ ce qui demande donc un traitement final si l'on souhaite uniquement $F(n)$.

Comparons les deux derniers algorithme pour une grande valeur $n=999$ : 
\begin{lstlisting}[frame = single]
print time.clock()
print fib2(999)
print time.clock()
print fib3Aux(999)[1]
print time.clock()	
\end{lstlisting}
\begin{lstlisting}[frame=single,language=bash]
air-de-raphael:Info Raphael$ python TP2.py
0.018006
26863810024485[...]
0.019236
26863810024485[...]
0.019283
\end{lstlisting}
Le second algorithme met $0.001230$ secondes et le troisième $0.000047$ secondes.
\end{document}














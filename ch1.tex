\documentclass{mybourbaki}
\titre{Algorithmique : tris naïfs}
\lstset{language=Python, numbers = left, frame =single}

\begin{document}
\section*{Introduction des notations}

\paragraph{Objectifs}On se fixe les objectifs suivant pour ce cours
\begin{itemize}
\item compter le nombre d'opérations effectué par des algorithmes ;
\item créer des programmes conçus pour toutes les valeurs d'un certain type.
\end{itemize}
L'évaluation de complexité est \textit{asymptotique}. Le nombre d'opérations est exprimé en fonction de la \og taille \fg{}\note{Par exemple : pour un entier $x\in \N$ ça peut être la longueur de son expression en décimal ; pour une liste cela peut être le nombre d'éléments de la liste, si $l$ est de longueur $n$ alors sa taille peut être : ${\rm taille}(l) = \abs{l} = \sum_{i=0}^{n-1}{\rm taille}(l[i])$} des entrées.

\definition{ 
Soient $f,g : \N\vers \R^+$. On dit que $f = \rO(g)$ si : \[\exists C \in \R^+, \exists n_0 \in \N, \forall n \geq n_0, \; f(n) \leq C\cdot g(n).\]
}{Notation $\rO()$}
Dans la suite, $f: \N \vers \R^+$ sera l'application qui à une entrée de taille $n$ associe le nombre d'opérations nécessaires.

\definition{ 
Soient $f,g : \N \vers \R^+$. On dit que $f = \Omega(g)$ si : \[ \exists C\in \R^+, \exists n_0 \in \N,\forall n\geq n_0 , \; f(n) \geq Cg(n). \]
}{Notation $\Omega()$}

\definition{ 
Soient $f,g : \N \vers \R^+$. On dit que $f = \Theta(g)$ si : \[ f = \rO(g) \et f = \Omega(g).\]
}{Notation $\Theta()$}

\paragraph{Exercice}Montrer que $f = \rO(g)$ lorsque :
\begin{enumerate}
\item $f(n) = n^{d}$ et $g(n) = n^{d+i}, i\geq 0$ ;
\item $f(n) = \log n$ et $g(n) = n^{\eps}, \; \eps> 0$ ;
\item $f(n) = n^{k}$ et $g(n) = r^{n}$ avec $r>1$ et $k\in \N$.
\end{enumerate}

\definition{ 
Voici les éléments comptés :
\begin{itemize}
\item le nombre d'affectations ;
\item le nombre de comparaisons ;
\item le nombre de divisions :
\item ...
\end{itemize}
}{Opérations dans les algorithmes}
\paragraph{Exemple}Le nombre de divisions effectuées par l'algorithme d'Euclide pour $a,b$ (avec $a>b$) est une fonction de $b$ : $\Theta(\log b)$.

\section{Permutations}
\paragraph{Objectifs}Faire un tour d'horizon des algorithmes de tris.
\paragraph{Opérations}Les opérations intéressantes pour les tris :
\begin{itemize}
\item comparaisons ;
\item affectation et échanges de variables.
\end{itemize}

\proposition{
Les propriétés suivantes indiquent que $\N$ est totalement ordonné avec $<$ : 
\[ 
\systeme{
\forall x\neq y, &\; x< y \ou y< x, \\
\forall x,y,z, &\; x<y \et y<z \implique < z .
}
 \]
}{$\N$ est totalement ordonné}

\definition{ 
\textit{Trier} un tableau ou une liste d'entiers : $T$ de longueur $n\in \N$ c'est trouver une permutation $\sigma \in S_n$ telle que : \[ \forall i \leq n-1, \; T[\sigma(i)] \leq T[\sigma(i+1)].\]
}{Trier}
\paragraph{Exemple}Pour $T=(10,9,7,6,8)$ on a : \[\sigma = \matrice{0 & 1 & 2 & 3 & 4 \\ 4 & 3 & 1 & 0 & 2}. \]

\lemme{ 
Pour tout $n\in \N^{*}$, toute permutation de $S_n$ peut s'écrire comme un produit de transpositions.
}{}

\section{Tri à bulle}
\subsection{Principe}
Le principe de cet algorithme est de faire \og remonter \fg{} les valeurs les plus grandes vers la fin de la liste jusqu'à ce que le tableau soit \textit{trié}.

\definition{ 
Soit $T$ un tableau.

On note sa longueur ${\rm longueur}(T) = \abs{T} = n$. On indice le tableau de $0$ à $n-1$.

Un tableau est \textit{trié} si : \[ \forall i \in \ens{0,\ldots,n-2}, \; T[i] \leq T[i+1].\]
}{}

On peut écrire un premier algorithme pour tester si le tableau est bien trié.
\begin{lstlisting}
function tableauTrie(T)
	n = longueur(T)
	for i from 0 to n-2 do
		if T[i] > T[i+1] then return false
\end{lstlisting}

\subsection{Principe détaillé du tri Bulle (Bubble sort)}
On fait dans l'ordre :
\begin{itemize}
\item un balayage successif de $T$ de gauche à droite ;
\item au $k$-ième balayage :
\begin{itemize}
\item on parcourt les $n-k$ valeurs les plus à gauche ;
\item si une position $j$ est telle que $T[j] >T[j+1]$ alors on échange $T[j]$ et $T[j+1]$.
\end{itemize}
\end{itemize}

\paragraph{\'Ecriture de l'algorithme}
Une première écriture donne :
\begin{lstlisting}
function triBulle(T)
	n = longueur(T)
	for i from n-1 to 0 do
		for j from 0 to i-1 do
			if T[j] > T[j+1] then
				T[j],T[j+1] = T[j+1],T[j]
	return T			
\end{lstlisting}
En tenant compte du fait que si un balayage entier est effectué sans modification alors c'est que le tableau est trié, on obtient :
\begin{lstlisting}
function triBullePlus(T)
	n = longueur(T)
	i = n-1
	repeat
		trie = true
		for j from 0 to i-1 do
			if T[j] > T[j+1] then
				T[j],T[j+1] = T[j+1],T[j]
				trie = false
		i = i-1	
	until trie = true
	return T	
\end{lstlisting}

\subsection{Preuve de correction}
On va montrer l'existence d'invariants.

\proposition{ Soit $k\in\ens{0,\ldots,n-1}$. Si $i = k$ alors :
\begin{enumerate}
\item $T[k+1] \leq T[k+2] \leq \ldots \leq T[n-1]$ ;
\item $\forall i < k+1, T[i] \leq T[k+1]$.
\end{enumerate}
}{Invariants}
\demonstration{ 
Si $n-1 -k = 0$ alors c'est bien vérifié.

Supposons $l$ itérations de l'algorithme et que la propriété est vraie pour $n-1-k =l$. Dans ce cas, à l'itération $l+1$ :
\begin{itemize}
\item La boucle interne met en position $n-1-l$ la plus grande des valeurs $v$ entre les positions $0$ et $n-1-l$ de $T$.
\item Par hypothèse de récurrence, $v\leq T[k+1]\leq T[k+2] \leq \ldots \leq T[n-1]$, on a donc : \[ T[k] \leq T[k+1] \leq \ldots \leq T[n-1] \et \forall i<k, T[i] \leq T[k].\]
\end{itemize}
}{Par induction/récurrence sur $n-1-k$}

\subsection{Analyse de complexité}On note $A$ un algorithme, $T$ une entrée de $A$ (ici un tableau) et $C_A(T)$ le nombre d'opérations\note{Ici : échanges, comparaisons et affectations.} que l'algorithme $A$ effectue sur l'entrée $T$.

\definition{ 
La \textit{complexité de $A$ dans le pire des cas} est la fonction : \[ \fonc{\cC_A}{\N}{\N}{n}{\max_{T : \abs{T} = n} C_A(T)}.\]

C'est-à-dire le nombre d'opération maximal sur une entrée de taille $n$.
}{}

\paragraph{Remarque}Soit $f:\N\vers \N$ :
\begin{itemize}
\item La complexité de $A$ est en $\rO(f(n))$  si $$\cC_A(n) \leq \rO(f(n)).$$
\item La complexité de $A$ est en $\Omega(f(n))$ s'il existe $n_0$, $T$ de taille $n\geq n_0$ et $c\in \R$ tel que \[ \cC_A(T) \geq c \cdot f(n).\]
\end{itemize}

\paragraph{Tri Bulle}On va essayer d'établir le nombre de comparaisons $C(n)$ :
\begin{itemize}
\item pour $n$ balayages :
\begin{itemize}
\item au $k$-ième balayage on effectue $n-k-1$ comparaisons.
\end{itemize}
\end{itemize}
On a alors : \[ C(n) = (n-1) + (n-2) +\ldots + 1 = \sum_{i=1}^{n-1}\frac{n(n-1)}{2}.\] Ainsi $C(n)$ est en $\rO(n^{2})$ et en $\Omega(n^{2})$. Le nombre d'échanges, $E(n)$ est en $\Omega(n^{2})$ avec comme pire des cas un tableau trié à l'envers.

\section{Tri par sélection}
\subsection{Principe}Le principe du tri par sélection est de chercher le maximum et de le placer à la fin. On itère avec $T$ entre les indices $0$ et $n-2$ (avec $n = \abs{T}$).

L'algorithme qui trouve le maximum :
\begin{lstlisting}
function posMax(T,p,f)
	k = p
	for i from p+1 to f do
		if T[i] > T[k] then
			k = i
	return k		
\end{lstlisting}

L'algorithme de tri en version itérative :
\begin{lstlisting}
function triSelectionIter(T,p,f)
	for i from f to p+1 do
		j = posMax(T,p,i)
		T[i],T[j] = T[j],T[i]
\end{lstlisting}
Et en version récursive :
\begin{lstlisting}
function triSelectionRec(T,p,f)
	if p < f then
		j = posMax(T,p,f)
		T[f],T[j] = T[j],T[f]
		triSelectionRec(T,p,f-1)
\end{lstlisting}

\subsection{Analyse de complexité (pour la version récursive)}
On compte le nombre d'échanges, $E(n)$, les affectations, $A(n)$, et les comparaisons, $C(n)$.

Par récurrence :
\begin{itemize}
\item Si $\abs{T}= 0,1$ alors $C(0) = E(0) = A(0) = 0$.
\item Si $\abs{T} = 2$ un appel à posMax, $C(1) = A(1) = 1$ et $E(1) = 1$.
\item Si $\abs{T} = n+1$ alors :
 \begin{enumerate}
 \item un appel à posMax avec sur $T$ : $n$ comparaisons et $n$ affectations (au plus) ;
 \item un échange ;
 \item un appel à triSelectRec sur le sous-tableau de taille $n-1$.
 \end{enumerate}
 On a donc $C(n+1) = n+C(n)$, $A(n+1) \leq n + C(n)$ et $E(n+1) \leq 1 + E(n)$.
\end{itemize}
Au final $C(n)$ et $A(n)$ sont en $\rO(n^{2})$ et en $\Omega(n^{2})$ (pour un tableau trié en ordre décroissant) et $E(n)$ est en $\rO(n)$ et en $\Omega(n)$.

\section{Recherche dans un tableau trié}
Rechercher est une activité essentielle en informatique. L'objectif est de le faire à moindre coût. On effectue un tri pour pouvoir rechercher efficacement.

\subsection{Recherche naïve}
Elle consiste à essayer toutes les possibilités :
\begin{lstlisting}
function rechercheBinaire(T,m,n,x)
	j = m
	while j < n+1 do
		if T[j] = x then return j
		else j = j+1
\end{lstlisting}

\subsection{Recherche dichotomique}
\paragraph{Principe}On compare l'élément $x$ à chercher avec la valeur du milieu du tableau. En fonction de la réponse on dirige la recherche suivante vers l'une des deux moitiés du tableau.

\begin{lstlisting}
function rechercheBinaire(T,l,u,x)
	if u < l then return
	m = int((l+u)/2)
	if T[m] = x then return m
	if T[m] > x then
		return rechercheBinaire(T,l,m-1,x)
	return rechercheBinaire(T,m+1,u,x)	
\end{lstlisting}

\subsection{Complexité de la recherche dichotomique}

On va compter le nombre de comparaisons, $C(n)$.
\begin{itemize}
\item $C(0) = 0$ ;
\item $C(1) =1$ (car on vérifie si c'est $x$ ou non) ;
\item si $u>l$, $m-l \leq u - m \leq m-l+1$ et alors $1+\max(m-l,u-m) \leq \frac{u-l+1}{2}$ et donc $C(n) \leq 1 + C(\lfloor \frac{n}{2}\rfloor)$.
\end{itemize}

\lemme{ 
La recherche dichotomique nécessite au plus \[ C(n) \leq \lfloor \log_2(n) + 1 \rfloor \leq \log_2 n +1\]
}{}
\paragraph{Preuve en exercice}Par récurrence sur $k = \lfloor \log_2 n \rfloor$. Indication : \[ C(n) \leq 1 + C(\lfloor n/2 \rfloor) \leq 1 + (\lfloor \log_2 n/2 \rfloor + 1) \leq 1 + \lfloor \log_2 n \rfloor.\]

\subsection{Preuve de correction}
Il faut :
\begin{enumerate}
\item vérifier que l'algorithme termine toujours (i.e. la taille des tableaux diminue strictement) ;
\item vérifier qu'il renvoie bien la bonne position si, et seulement si, $x$ est dans le tableau.
\end{enumerate}

\end{document}





























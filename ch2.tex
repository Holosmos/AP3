\documentclass{mybourbaki}
\titre{Tri fusion (\og Merge Sort \fg{})}
\lstset{language=Python, numbers = left, frame =single}
\begin{document}

\section*{Introduction}
\subsection*{Diviser pour régner}

C'est algorithme est une illustration du principe algorithmique \og diviser pour régner\fg{}. 

La résolution se fait en trois étapes :
\begin{enumerate}
\item division du problème en un certain nombre de sous-problèmes ;
\item résolution des sous-problèmes récursivement ;
\item assemblage des solutions des sous-problèmes.
\end{enumerate}

\subsection*{Dans le tri fusion}
Si $A$ est un tableau de longueur $n$.
\begin{enumerate}
\item On divise $A$ en deux tableaux $A_1$ et $A_2$ de longueur divisée par deux :
\item on trie $A_1$ et $A_2$ ;
\item on fusionne les deux tableaux $A_1,A_2$ triés.
\end{enumerate}

\section{Fusion de deux tableaux triés}

\subsection{Principe}

On se donne deux tableaux triés $A,B$. L'objectif est d'obtenir $C$ tel que $\abs{C} = \abs{A} + \abs{B}$ et tel que $C$ soit la fusion triée de $A$ et $B$.

\paragraph{Exemple}Si $A = (1,2,3,7,9,11)$ et $B = (2,4,5,13)$ alors $C = (1,2,2,3,4,5,7,9,11,13)$.

\paragraph{Principe}Si $C$ est indicé par $k$ allant de $0$ à $n+m-1$ (avec $n = \abs{A},m=\abs{B}$), alors :
\begin{enumerate}
\item on maintient $i$ et $j$ positions respectives dans $A$ et $B$ et on compare $A[i]$ et $B[j]$ ;
\item on insère le plus petit des deux dans la liste $C$ et on augmente le compteur concerné de $1$ ;
\item cas spécial quand l'un des deux tableaux est déjà trié (auquel cas aucune comparaison n'est nécessaire, on adjoint simplement).
\end{enumerate}

\begin{lstlisting}
function Merge(A,B)
	n = longueur(A), m = longueur(B)
	i = 0, j = 0
	for k from 0 to n+m-1 do
		if i > n-1 then
			C[k] = B[j], j = j+1
			continue
		if j > m-1 then
			C[k] = A[i], i = i+1
			continue
		if A[i] > B[i] then
			C[k] = B[j], j = j+1
		else
			C[k] = A[i], i = i+1
	return C					
\end{lstlisting}

\subsection{Preuve de correction}

\proposition{ 
Soit $[[k]],[[i]],[[j]]$ les valeurs respectives des variables $k,i,j$ avec $[[k]] = [[i]] + [[j]]$. À l'issue d'une itération de la boucle \textit{for}, les indices de $0$ à $[[k]] - 1$ de $C$ contiennent les éléments de :
\begin{itemize}
\item $A$ entre $0$ et $[[i]] - 1$,
\item $B$ entre$ 0$ et $[[j]] - 1$,
\end{itemize}
et triés dans l'ordre croissant.
}{Invariant}
\demonstration{ 
On procède par récurrence sur $k$ :
\begin{itemize}
\item Pour $[[k]]=1$ c'est clair.
\item Si $[[k]]-1$ indices sont vérifiés, alors $[[k]]$-ième est : \[ C[k] = \min \ens{A[i],B[j]} = \min \ens{A[h] : i\leq h \leq n-1, B[k] : j\leq k \leq n-1}.\] La propriété est bien vérifiée.
\end{itemize}
}{}

\subsection{Complexité}
On évalue le nombre de comparaisons.
\begin{itemize}
\item On effectue au plus trois comparaisons par itération, si $C(n)$ est le nombre de comparaisons : \[ C(n) = \rO(n+m), \; C(n) = \Omega(n+m).\]
\end{itemize}


\section{Algorithme pour le tri fusion}
\subsection{Algorithme}
\begin{lstlisting}
function MergeSort(A)
	n = longueur(A)
	if n <= 0 then return A
	m = int((n-1)/2)
	A1 = A[0 : m-1]
	A2 = A[m : n-1]
	A1 = MergeSort(A1)
	A2 = MergeSort(A2)
	A = Merge(A1,A2)
	return A
\end{lstlisting}

\paragraph{Illustration}Par exemple :
\begin{align*}
&\matrice{7 & 3 & 4 & 8 & 2 & \vline & 15 & 5 & 6} \\
\to & \matrice{7 & 3 & \vline & 4 & 8} \; \matrice{2 & 15 & \vline & 5 & 6} \\
\to & \matrice{7 & 3} \; \matrice{4 & 8} \; \matrice{2 & 15} \matrice{5 & 6} \\
\to & \matrice{7} \; \matrice{3} \; \matrice{4} \; \matrice{8} \; \matrice{2} \; \matrice{15} \; \matrice{5} \; \matrice{6} \; \; \text{(fin division)} \\
\to & \matrice{3 & 7} \; \matrice{4 & 8} \; \matrice{2 & 15} \; \matrice{5 & 6} \\
\to &\matrice{3 & 4 &7 & 8} \; \matrice{2 & 5 & 6 & 15} \\
\to &\matrice{2 & 3 & 4 & 5 & 6 & 7 & 8 & 15} \; \; \text{(fin fusion)}
\end{align*}

\subsection{Complexité}
Soit $C(n)$ le nombre de comparaisons pour l'algorithme de tri fusion. Alors \[ C(n) = C(\lfloor n/2 \rfloor) + C(\lceil n/2 \rceil) + n-1\note{Avec un algorithme plus performant pour la fusion on peut avoir un coût de $n-1$.}\]

\theoreme{ 
Si $A$ est un tableau de longueur $n$, alors le nombre de comparaisons, $C(n)$, effectuées par le tri fusion est de $\rO(n\log n)$ dans le pire des cas.
}{}
C'est un cas particulier du \og \textsc{Master} Theorem \fg{}.

\demonstration{ 
On cherche à majorer la suite $(t_n)$ définie par $t_0 = t_1 = 0$ et \[t_n \leq n-1 + t_{\lfloor n/2\rfloor} + t_{\lceil n/2 \rceil}. \]

On suppose d'abord que $n=2^{k}$ pour un certain $k\in \N$. On étudie $(u_k)$ telle que $u_0=u_1=0$ et : \[ u_{k+1} = 2^{k+1}-1 + 2u_k.\]On peut montrer que le terme général de cette suite est : \[ u_k = (k-1)2^{k}+1.\]On a donc \[ t_{2^{k}} \leq u_k = (k-1)2^{k}+1.\]

Soient $n\in \N^{*}$ et $k\in \N^{*}$ tels que $$2^{k}< n \leq 2^{k+1}-1.$$ Montrons que $t_n \leq u_{k+1}$ :
\begin{itemize}
\item pour $n=2$ c'est vérifié, $2^{0}< n \leq 2^{1}$ et $t_2 = u_1$ ;
\item supposons que c'est montré pour $n+1$ tel que $2^{k}<n+1 \leq 2^{k+1}$, on a \[ t_{n+1} \leq n + t_{\lfloor n/2 \rfloor} + t_{\lceil n/2 \rceil}\] or $\lceil (n+1)/2 \rceil \leq 2^{k}\leq n$ et donc \[ t_{n+1} \leq 2^{k+1}-1 + u_k+ u_k\]et donc \[ t_{n+1}\leq u_{k+1}.\]
\end{itemize}

Ainsi, si $k$ est tel que $2^{k}<n\leq 2^{k+1}$ alors \[ t_n \leq u_{k+1} = k2^{k+1}+1\]c'est-à-dire : \[ t_n < (2n)\log_2(n) + 1.\]
}{}

\end{document}













